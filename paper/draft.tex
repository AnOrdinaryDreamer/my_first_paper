\documentclass{article}
\usepackage{arxiv}

\usepackage[utf8]{inputenc}
\usepackage[english, russian]{babel}
\usepackage[T1]{fontenc}
\usepackage{url}
\usepackage{booktabs}
\usepackage{amsfonts}
\usepackage{nicefrac}
\usepackage{microtype}
\usepackage{lipsum}
\usepackage{graphicx}
\usepackage{natbib}
\usepackage{doi}



\title{Методы пространственной интерполяции в задаче пространственной интерполяции для оценки характеристик лесного массива}

\author{Гуров С.И.\\
	МГУ им. М.В. Ломоносова\\
	факультет ВМК\\
        119991, Москва, Ленинские горы, д.1, стр. 52
	\texttt{<sgur@cs.msu.ru>} \\
	\And
	  Илларионова С.В.\\
	  Сколковский институт науки и технологий\\
	121205, г. Москва, тер. Сколково Инновационного Центра, б-р Большой, д.30 стр. 1\\
	\texttt{s.illarionova@skoltech.ru} \\
    \And
    Глазырина С.Е. \\
	МГУ им. М.В. Ломоносова\\
	факультет ВМК\\
        119991, Москва, Ленинские горы, д.1, стр. 52
	\texttt{sv.glazyr@gmail.com} \\
}
\date{}

\renewcommand{\shorttitle}{\textit{arXiv} Template}

%%% Add PDF metadata to help others organize their library
%%% Once the PDF is generated, you can check the metadata with
%%% $ pdfinfo template.pdf
\hypersetup{
pdftitle={My first paper Glazyrina},
pdfkeywords={timber stock estimation, kriging},
}

\begin{document}
\maketitle

\begin{abstract}
	Текст аннотации
    
\end{abstract}


\keywords{пространственная интерполяция \and оценка запасов древесины}

\section{Introduction}
Написать, почему важно оценка состояния лесов важна.
Запас древесины - один из ключевых факторов в управлении лесным хозяйством. Обычно он оценивается в ходе полевых измерений.

Однако, из-за высокой стоимости и низкой доступности удалённых от дорог и рек лесных участков, количество проводимых во время таксации леса измерений остаётся ограниченным, а измерения - разреженными. Наша задача состоит в картировании запасов древесины в местности на основе этих немногих измерений. Для решения проблемы недостатка данных, многие исследователи, не только в задачах лесоводства, но и других задачах оценки окружающей среды, прибегают к дистанционному зондированию Земли. Использование таких признаков позволяет заметно повысить качество оценки.

To address this problem, satellite remote sensing and spatial interpolation/extrapolation technologies have been commonly used in past research \citep{}. But now ...

\section{Обзор литературы}

В последние годы с ростом доступности дистанционного зондирования для оценки запасов древесины помимо данных лесного кадастра, имеющихся лишь для малого числа точек области из-за высоких трудозатрат связанных с их получением, в качестве дополнительных данных стали использоваться спутниковые снимки и данные, полученные с помощью лидара \citep{maack2016modelling}. Как показывает практика, использование этих данных, даже без учёта пространственных переменных, уже позволяет добиться хорошей точности в задаче оценки объемов древесины. В статье \cite{sanchez2019growing} описан ряд признаков, получаемых на основе этих данных и дающих наибольший вклад в решение этой задачи.

При рассмотрении данной задачи естественным кажется предположение о пространственной непрерывности целевой переменной. Оно является базовым для всех методов пространственной интерполяции. Простейшие детерминистические методы пространственной интерполяции широко использовавшиеся в 90-е годы, такие как метод радиальных базисных функций или полиномиальные интерполяторы, позволяют использовать информацию о координатах точки измерения для оценки целевой переменной в этой точке \citep{kanushin2023review}. Если усилить постановку задачи предположением о пространственной стационарности целевой переменной, то к рассмотрению доступных методов добавляется группа стохастических методов типа кригинга - в зависимости от предположений о поведении среднего значения переменной, можно использовать как простой (среднее значение известно и постоянно), обыкновенный (среднее значение неизвестно и постоянно во всей области оценивания) или универсальный (среднее значение неизвестно и непрерывно изменяется в области оценивания) кригинг \citep{geostatics_book}. Их преимущество состоит в поиске не только оценки целевой велчины, но и оценить дисперсию ошибки, но предположение о стационарности существенно \citep{krukova2018spatialinterpolation}. Предположения можно ослабить, используя метод для оценивания значений в некоторой ограниченной подобласти (например, учитывать лишь точки находящиеся от точки, в которой производится оценивание, не далее заданного порога), а не как глобальный интерполятор. Но ни один из этих методов не способен учесть информацию о дополнительных признаках. Для решения этой проблемы можно применить кригинг с внешним дрейфом, который использует дополнительные данные измерений коррелированной переменной в качестве модели тренда, и потому позволяет достаточно точно оценить тренд при наличии данных дополнительной тренд-переменной во всех точках оценивания \citep{hudson1994mapping}. Другой способ учесть дополнительные признаки - использование кокригинга - обобщение кригинга на случай многих переменных, предполагающий, что между ними есть пространственная корреляция \cite{wackernagel1995multivariate}.

В последние годы для решения подобных задач чаще стали использоваться различные нейросетевые архитектуры. При анализе пространственно-временных зависимостей достаточно удачным оказался подход из комбинации LSTM и некоторого блока Geo-Layer (реализации которого различны в разных работах), позволяющего учесть пространственную корреляцию \citep{sanchez2019growing, MA2019117729, maack2016modelling}. Кроме этого, использованные в статье \cite{MA2019117729} методы машинного обучения, в частности случайный лес, показали лучший по сравнению с традиционными методами геостатистики и нейросетевыми архитектурами результат при рассмотрении пространственно скоррелированных данных. В статье \citep{LIU2011568} предлагается архитектура нейронной сети, использующая радиальные базисные функции. Для экспериментов, проведённых в этой статье, такая нейронная сеть оказалась лучшим пространственным интерполятором, по сравнению с кригингом, так как не требует выполнения множества предположений кригинга, но при этом получаемая функция координат не является непрерывной. В статье \cite{maack2016modelling} предлагается сравнение GAM и случайного леса на задаче оценки запасов древесины. Полученные ими результаты говорят о более высокой эффективности применения GAM для больших и разнородных участков, что не так интересно в нашем случае. При этом в ряде других исследования применение ансамбля из деревьев давало более высокий по сравнения с GAM результат.

Ведутся исследования в области разработки новых методов пространственной интерполяции. Зачастую они являются комбинацией известных детерминистических алгоритмов и методов машинного обучения. Например, в статье \cite{kataev2013classification} объекты обучающей выборки сначала разбиваются на кластеры со схожими свойствами, после чего для интерполятор при оценке учитывал только объекты из того кластера, в котором находится точка оценивания.

Решаемую задачу можно переформулировать в терминах обучения с частичным привлечением учителя. Исследования в этой области ведутся уже достаточно давно и существует целая иерархия методов регрессии в задаче частичного обучения \citep{Kostopoulos2018, Xu2021SSR}. 



\bibliographystyle{unsrtnat}
\bibliography{references}

\end{document}
